\chapter{Introduction}
\label{chap:intro}

% In today's world, hyperscale cloud providers, such as Microsoft Azure and Amazon Web Services, are used extensively. These companies provide serverless functions, which run for short bursts and close immediately afterwards. In order to do this efficiently, the sandbox in which the code is executed should be able to quickly spin up, so the cold start problem can be minimized. Existing services like Docker and K8S are mostly used for this, but are not ideal: while being more performant than full blown operating systems, there cold boot is still problematic. A new standard, the WebAssembly System Interface (WASI), has been created which should greatly mitigate this problem.
% 
% \acrfull{Wasm} originally started as a way for browsers to execute code written in a language other than Javascript. It allows code from any supported programming language to be compiled to machine instructions for a virtual CPU, and for these instructions to be efficiently executed in the browser sandbox. This has proven to be successful and has opened numerous possibilities for web apps that weren't possible before. A great example of this is the new Google Earth website, which uses WebAssembly underneath, which allows for performant rendering of the Earth.
% 
% This model of running code on a lightweight virtual CPU seen interest outside of the browser, mainly for servers, low power devices (IOT), and so on. There, the WebAssembly System Interface could replace Docker, and provide performance improvements, power and storage savings.
% 
% WASI is currently still in development, and has a long way to go to get stable. One of the problems WASI is currently facing is the lack of APIs available to interface with the system. One of the APIs that is currently missing is the one to interact with USB devices. Having such an API would be useful for servers, cars, IOT devices, etc. The goal of this master's thesis is to create an USB proposal for expanding the WASI, and eventually add support for USB to WASI. With access control, users can decide which devices a program running in WASM can access.


% V2 TEKST

Updating software of \acrfull{IoT} devices is currently difficult. As the devices are home appliances, they are often used for longer periods of time, compared to other tech such as smartphones. As a consequence, these devices need longer software support. In reality, a lot of devices stop getting updates after a few years, making them vulnerable to cyberattacks. When an \acrshort{IoT} device gets hacked, it can be used as part of a botnet, or worse, send sensitive info (such as a video feed) to the attacker. There are multiple causes to why this software support period is short: no standardized updating mechanisms, special hardware and compilers, etc. \cite{wasi_iot}

In Web browsers, \acrfull{Wasm} is used to run compiled code in a lightweight virtual machine. Code gets compiled to \acrshort{Wasm}, but can be run on any platform that supports \acrshort{Wasm}, making it cross-platform. This model can also be applied outside the browser context and can solve some of the problems of native code. However, \acrshort{Wasm} is primarily targeted towards browser applications, so most tooling is only available in the browser. Calling APIs to the outside world is also cumbersome in Wasm, as it has a limited featureset to call APIs. Also, no low-level libraries exist for controlling hardware devices, which would be required for \acrshort{IoT} devices. To solve this, a new specification was created: the \acrfull{WASI}

\acrshort{WASI} is a collection of API specifications that can be used by \acrshort{Wasm} applications outside the browser. Thanks to \acrfull{WIT} and the canonical ABI, these APIs can provide a more developer-friendly interface compared to Wasm. These APIs can still be used by any programming language that offers support for WASI. Through the use of generated bindings, the API defined in \acrfull{WIT} can be converted to a language-tailored variant, integrating well with features the language offers. Because the APIs act as a layer between the native APIs and the application, access control can be applied to control what an application can access. 

With \acrshort{WASI}, standardized APIs can be created that can be used to talk and control hardware. As the APIs are not device specific anymore and no compiling to specific architectures is needed anymore, updating software for \acrshort{IoT} devices has become a lot easier.

\section*{Problem Statement}

The \acrshort{WASI} standard is still in development and currently lacks an API to interact with USB devices. Having an API for USB devices is crucial for \acrshort{IoT} devices as a lot of hardware, such as cameras, will oftentimes communicate over USB.

\section*{Goal}

The goal of this master's thesis is to research and propose a new USB API for \acrshort{WASI}. Special attention will be given to access control, portability and performance.

\section*{Research Questions}

\begin{itemize}

\item What aspects of the USB interface can benefit of access control?

\item In what way can a WASI API be created that makes porting code easy?

\item What is the performance impact of this API compared to native code?

\end{itemize}
